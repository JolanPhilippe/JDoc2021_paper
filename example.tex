\begin{multicols}{2}


Social network vendors often provide specific development platforms, used by developers to
build apps that extend the functionality of the social network. Some networks are associated
with marketplaces where developers can publish such apps, and end-users can buy them. 
Development platforms typically include APIs that allow analyzing and updating the social
network graph. 

As a running example for this paper, we consider a scenario where a vendor adds a LCDP to allow
end-users (also called \emph{citizen developers} in the LCDP jargon) to implement their own apps.
Such LCDP could include a `What you see is what you get editor' for the app user-interface, and a visual workflow for
the behavioral part. In particular, the LCDPs would need to provide mechanisms, at the highest
possible level of abstraction, to express queries and updates on the social graph.

In Figure~\ref{fig:ttc_mm} we show the simple metamodel for the social graph that we will use in
the paper. The metamodel has been originally proposed at the Transformation Tool Contest (TTC)
2018~\cite{Garcia2018:TTC}, and used to express benchmarks for model query and transformation
tools. In this metamodel, two main entities belong to a !SocialNetwork!. First, the !Post!s and
the !Comment!s that represent the !Submission!s, and second, the !User!s. Each !Comment! is
written by a !User!, and is necessarily attached to a !Submission! (either a !Post! or another
!Comment!). Besides commenting, the !User!s can also like !Submission!s. 

As an example, in this paper we focus on one particular query, also defined in TTC2018: the
extraction of the three most debated posts in the social network. To measure how debated is the
post, we associate it with a numeric score. The LCDP will have to provide simple and efficient
means to define and compute this score. We suppose the vendor to include a declarative query
language for expressing such computation on the social graph, and storing scores as a derived
properties of the graph (i.e. new properties of the social graph that are computed on demand
from other information in the graph). 

\end{multicols}

